\newpage
\section{Обзор работ по тематике исследования}
\label{subsec:Rewiev}
Исследования, связанные с анализом данных, возникающих при использовании GitHub и разработке продуктов на этой платформе, представляют собой популярную тему для научных исследований.

Так, авторы статьи~\cite{abs-2308-14211} обсуждают проблему классификации отзывов о приложениях, которые содержат важную информацию о потребностях пользователей. Исследователи предлагают подход для создания более обобщенной модели, используя информацию из системы отслеживания задач GitHub, которая содержит ценные данные о потребностях пользователей. После проведения экспериментов, они показывают, что использование размеченных задач из GitHub может улучшить точность и полноту классификации отзывов, особенно для отчетов о багах и запросов на новые функции. В другой статье~\cite{RoccoRSNR23} поднимается важность репозиториев программного обеспечения для управления проектами, включая исходный код, документацию и отчеты об ошибках. Особое внимание уделяется платформе GitHub, которая для помощи разработчикам в поиске подходящих артефактов использует темы (topics), являющимися короткими текстами, присваиваемыми хранимым артефактам. Однако неправильное присвоение тем может негативно сказаться на популярности репозитория. 

Закария Альшара и другие авторы в своей статье~\cite{AlsharaSSS23} рассматривают проблему управления задачами (issues) на платформе GitHub, особенно в случае быстрого роста числа создаваемых задач. Для помощи разработчикам в обработке задач существуют внешние участники, которые исправляют задачи, создавая pull-запросы (Pull Requests, они же PR). Однако часто такие PR не связываются с соответствующими задачами (issues), что затрудняет управление проектом. В статье предлагается использование моделей машинного обучения (ML) для автоматического восстановления связей между PR и задачами на GitHub. Установление связей между PR и задачами ценно, так как это помогает улучшить управление разработкой и обслуживанием проектов, что влияет на популярность и развитие проекта в дальнейшем.

В исследовании~\cite{RamasamySBB23} исследует, как проводится кодирование в области науки о данных на GitHub. Авторы анализируют, как данные ученые переходят между разными этапами работы с данными. Результаты исследования показывают, что кодирование имеет определенные паттерны. Кроме того, авторы попытались обучить модели машинного обучения для предсказания этапов работы с данными и достигли точности примерно 71\%.

О популярных проектах говорят в статье Джесси Айала и ее коллеги~\cite{AyalaG23}, а именно о важности использования непрерывной интеграции и поставки (CI/CD) и политики безопасности в известных и пользующихся интересом проектах с открытым исходным кодом, особенно на GitHub. Исследование показало, что многие проекты не активно используют эти возможности, и призывает управляющих таких проектов уделить им больше внимания для предотвращения уязвимостей. А в другом исследовании~\cite{PuhlfurssMM22} фокусируются на том, как документируется информация о функциональных возможностях программного обеспечения в проектах на GitHub и связана ли она с исходным кодом. Авторы провели анализ 25 популярных репозиториев на GitHub и обнаружили, что хотя документация о функциональности часто присутствует в различных текстовых файлах, она часто неструктурированна, и связь с исходным кодом редко устанавливается, что может привести к затруднениям в его поддержке на долгосрочной перспективе. 

Кроме того, существует статья~\cite{CasalnuovoSRR17}, рассматривающая инструмент GitcProc, который предназначен для анализа проектов на GitHub. Этот инструмент позволяет извлекать информацию о разработке, включая исходный код и историю исправления ошибок. GitcProc может отслеживать изменения в исходном коде и связывать их с функциями с минимальными настройками. Он успешно работает с проектами на разных языках программирования, обнаруживая исправления ошибок и контекст изменений в коде. Помимо этого, стоит также рассмотреть следующую работу~\cite{SollV17}, где авторы обсуждают задачу классификации репозиториев на GitHub, которая представляет собой сложную задачу. Они представляют алгоритм ClassifyHub, основанный на методах ансамблирования, разработанный для соревнования InformatiCup 2017. Этот алгоритм успешно решает задачу классификации с высокой точностью и полнотой, что может быть полезно для различных приложений, таких как рекомендательные системы.

Самой приближенной к поставленной задаче статьей является <<A Cross-Repository Model for Predicting Popularity in GitHub>>~\cite{abs-1902-05216}, в которой рассматривается создание модели для прогнозирования популярности репозиториев на GitHub, используя данные из разных репозиториев. Модель, основанная на рекуррентной нейронной сети LSTM, позволяет более точно предсказывать популярность, чем стандартные методы прогнозирования временных рядов на основе данных из одного репозитория. 

Кроме неё, есть также исследование~\cite{HanDXWY19}, в котором предлагают метод для прогнозирования популярности проектов на GitHub. Он использует 35 признаков, извлеченных из GitHub и Stack Overflow, чтобы классифицировать проекты как популярные или нет. Модель, основанная на случайном лесе, достигает высокой точности значительно превосходит существующие методы. Основными признаками для определения популярности оказались количество веток, количество открытых задач и количество участников проекта.