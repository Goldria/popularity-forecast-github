\newpage
\section{Анализ предметной области}
\label{sec:Background}
GitHub -- это веб-платформа, основанная на системе контроля версий Git, предназначенная для хранения, управления и совместной разработки программного обеспечения. Он позволяет программистам хранить свой код в репозиториях, отслеживать изменения в коде с помощью коммитов, веток и слияний, а также управлять задачами и проектами с помощью системы отслеживания ошибок и запросов на слияние. 

GitHub обеспечивает возможность совместной работы над проектами, позволяя разработчикам работать параллельно над различными частями кода, комментировать изменения, предлагать исправления и обсуждать детали реализации. Платформа также предоставляет инструменты для управления версиями проекта, позволяя возвращаться к предыдущим версиям кода и отслеживать историю изменений. GitHub широко используется в мире разработки программного обеспечения для совместной работы над открытыми и закрытыми проектами, обмена кодом и облегчения процесса разработки и сотрудничества.

Для создания классификации проектов на GitHub необходимо понять, по каким критериям можно выстроить и определить <<рейтинг>> популярных репозиториев. Для работы с информацией о проектах на GitHub будет использоваться существующий датасет, загруженный в ClickHouse, кроме того, он содержит достаточно много данных, которые могут быть полезны и иметь достаточно большой вес в возможности классифицировать репозиторий. 
 
 Выделим основные параметры относительно перечисленных ранее:
 \begin{itemizecustom}
     \item Звезды (Stars). Количество звезд  репозитория указывает на его популярность. Звезды представляют интерес и поддержку сообщества разработчиков и пользователей. Большое количество звезд может привлечь внимание новых разработчиков и повысить репутацию репозитория.
     \item Форки (Forks). Количество форков показывает, сколько раз репозиторий был скопирован другими разработчиками для дальнейшей работы. Большое количество форков может означать активное участие сообщества, что не редко приводит к активному развитию.
    \item Проблемы (Issues). Количество задач отражает активность сообщества в обнаружении и решении проблем и задач. Активные задачи могут привлечь новых участников и улучшить качество проекта.
    \item Коммиты (Commits). Количество коммитов указывает на активность разработчиков в репозитории. Активность и последовательность коммитов важны для развития и поддержания проекта.
    \item Пулл-реквесты (Pull Requests). Количество запросов на слияние отражает вклад и сотрудничество участников проекта. Пулл-реквесты представляют собой важный инструмент совместной разработки и улучшения кода.
    \item Количество пользователей, делающих коммиты. Когда множество разработчиков активно участвует в проекте, это может служить подтверждением его ценности и качества, может создать доверие у новых пользователей и их убеждение в том, что проект стоит внимания.
 \end{itemizecustom}

\subsection{Теоретический базис}

\label{sec:Models}

Существует разнообразие моделей машинного обучения, каждая из которых ориентирована на решение конкретных типов задач:
\begin{itemizecustom}
    \item     Регрессионные модели: Они используются для прогнозирования численных значений или характеристик объектов.

    \item     Модели классификации: Они предсказывают принадлежность объекта к определенной категории на основе заданных параметров. 
\end{itemizecustom}

Регрессионные модели ориентированы на прогнозирование количественных значений, в то время как модели классификации сосредоточены на определении принадлежности объекта к определенным категориям или классам на основе входных параметров. 

Рассмотрим модели, которые будут использованы для работы.

\textbf{Деревья решений}

Деревья решений -- это графические структуры, используемые в машинном обучении для принятия решений на основе условий или правил, представленных в виде дерева. Они представляют собой модель, которая аппроксимирует входные данные с помощью последовательности решений, ведущих к конечным выводам или предсказаниям.

Деревья решений могут использоваться как для задач классификации, когда необходимо отнести объект к одной из категорий, так и для задач регрессии, когда нужно предсказать численное значение. Они предоставляют простой и понятный способ моделирования данных, однако большие деревья могут быть склонны к переобучению.

\textbf{Случайный лес}

Случайный лес (Random Forest) -- это вид ансамбля деревьев, где каждое дерево строится независимо друг от друга. В процессе обучения каждое дерево получает подмножество данных (выбирается случайным образом из общего набора данных), и на основе этого подмножества строится свое дерево решений. 

\textbf{Градиентный бустинг}

Градиентный бустинг (Gradient Boosting) -- это метод построения ансамбля деревьев последовательно, каждое новое дерево исправляет ошибки предыдущего. На каждом шаге новое дерево строится таким образом, чтобы минимизировать остатки (разницу между предсказаниями модели и реальными значениями). При обучении каждое следующее дерево фокусируется на тех объектах, на которых предыдущие модели ошиблись больше всего. 

\textbf{Наивный Байесовский классификатор}

Наивный Байесовский классификатор -- это простая вероятностная модель, основанная на теореме Байеса, которая используется для решения задач классификации. Он основан на предположении о независимости между признаками. 

\textbf{AdaBoost (Adaptive Boosting)}

Адаптивный бустинг (AdaBoost) -- алгоритм машинного обучения, используемый для улучшения производительности других алгоритмов классификации. Он работает путем последовательного обучения слабых моделей классификации на различных взвешенных подмножествах обучающих данных.
