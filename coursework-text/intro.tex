\newpage
\newcounter{biconvert}
\setcounter{biconvert}{\totvalue{bibitems}}
\regtotcounter{chapter}
\sectionnonumber{Введение}
\subsection*{Актуальность}
С развитием технологий в сфере информационных технологий и программного обеспечения объем данных, генерируемых и публикуемых в репозиториях, таких как GitHub, Bitbucket, Gogs и др, продолжает стремительно расти. GitHub является крупнейшей платформой для хостинга и совместной разработки кода, предоставляющей доступ к миллионам проектов, созданных сообществом разработчиков со всего мира. Столь огромный объем репозиториев создает необходимость в разработке эффективных методов и инструментов для их классификации и анализа.

Одним из ключевых аспектов, определяющих значимость проекта, является его популярность, которая может влиять на привлекательность проекта для потенциальных участников, уровень вовлеченности сообщества разработчиков, а также на его долгосрочную жизнеспособность. Поэтому разработка системы, способной автоматически классифицировать проекты GitHub по их популярности, имеет огромное значение для исследования и индустрии разработки программного обеспечения.

В настоящее время отсутствуют универсальные и эффективные методы оценки популярности проектов на платформе GitHub, а существующие подходы могут предоставить лишь ограниченное представление проектах. В свете этой проблемы разработка системы для классификации проектов GitHub по популярности становится значимой задачей. Такая система должна учитывать разные факторы, влияющие на популярность проекта, и применять современные методы машинного обучения для создания более точного и всестороннего представления о популярности проектов на GitHub.

\subsection*{Постановка задачи}

Целью данной курсовой работы является разработка системы для классификации проектов на платформе GitHub по уровню их популярности. Для достижения этой цели были сформулированы следующие задачи.

\begin{enumerateparen}
    \item Выполнить анализ предметной области и провести обзор существующих решений.

    \item Выбрать репрезентативный набор признаков и подготовить данные по существующим репозиториям GitHub.

   \item Исследовать различные модели машинного обучения и выбрать наиболее эффективные.

    \item Разработать приложение, которое будет классифицировать проекты по нескольким уровням на основе выбранной модели.
\end{enumerateparen}

\vspace{0.5em}
\subsection*{Структура и содержание работы}

Курсовая работа состоит из введения, четырех разделов, заключения и списка литературы. Объем текста составляет \numplural{\getpagerefnumber{LastPage}}{страниц}{страницу}{страницы}, объем списка литературы~-- \numplural{\arabic{biconvert}}{наименований}{наименование}{наименования}.

В первом разделе описывается предметная область, а также проводится анализ существующих аналогов и методов, решающих поставленные задачи курсовой работы, и теоретический базис относительно моделей обучения.

Во втором разделе описываются функциональные и нефункциональные требования к разрабатываемому приложению. Приведены варианты использования программы, архитектура приложения, его компоненты и макет пользовательского интерфейса.

В третьем разделе содержит описание программных средств, используемых в процессе реализации системы, описание реализации определения класса популярности проекта (звезд), а также пользовательского интерфейса.

В четвертом разделе представлены функциональное тестирование приложения, а также и оценка точности полученных результатов согласно метрикам машинного обучения.

В заключении представлены основные результаты выполненной работы и направления, в которых возможны дальнейшие исследования.

В приложениях содержатся спецификация диаграммы вариантов использования системы, интерфейсы и листинги основных модулей, а также результаты матрицы ошибок моделей и сравнение метрик в соответствии со временем выполнения.
