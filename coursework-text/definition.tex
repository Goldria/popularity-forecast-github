\newpage
\section{Анализ предметной области}
\label{sec:Definition}

 Для создания классификации проектов на GitHub необходимо понять, по каким критериям мы сможем выстроить и определить определенный <<рейтинг>> репозиториев. Для работы с информацией о проектах на GitHub будет использоваться существующий датасет, загруженный в ClickHouse, кроме того, он содержит достаточно много данных, которые могут быть полезны и иметь достаточно большой вес в возможности классифицировать репозиторий. Выделим основные параметры:
 \begin{itemize}
     \item Звёзды (Stars). Количество звёзд  репозитория указывает на его популярность. Звёзды представляют интерес и поддержку сообщества разработчиков и пользователей. Большое количество звёзд может привлечь внимание новых разработчиков и повысить репутацию репозитория.
     \item Форки (Forks). Количество форков показывает, сколько раз репозиторий был скопирован другими разработчиками для дальнейшей работы. Большое количество форков может означать активное участие сообщества, что не редко приводит к активному развитию.
    \item Авторы (Contributors). Количество участников указывает на количество разработчиков, внесших свой вклад в репозиторий. Большое количество участников может говорить о широком интересе и активности в проекте.
    \item Проблемы (Issues). Количество задач отражает активность сообщества в обнаружении и решении проблем и задач. Активные задачи могут привлечь новых участников и улучшить качество проекта.
    \item Коммиты (Commits). Количество коммитов указывает на активность разработчиков в репозитории, а указание автора может выделить ведущего разработчика. Активность и последовательность коммитов важны для развития и поддержания проекта.
    \item Пулл-реквесты (Pull Requests). Количество запросов на слияние (pull requests) отражает вклад и сотрудничество участников проекта. Пулл-реквесты представляют собой важный инструмент совместной разработки и улучшения кода.
 \end{itemize}

 \textit{Следующий текст будет добавлен в другой раздел, здесь он сейчас размещен в рамках продолжения темы с основными параметрами репозиториев}:

\begin{lstlisting}[language=SQL, caption= Запрос на получение данных из датасета]
SELECT
    repo_name,
    SUM(event_type = 'ForkEvent') AS forks,
    SUM(event_type = 'WatchEvent') AS stars,
    count() AS contributors,
    count() AS issues,
    count() AS commits,
    count() AS pull_requests
FROM github_events
WHERE event_type IN ('ForkEvent', 'WatchEvent', 'PullRequestEvent', 'IssuesEvent', 'CommitCommentEvent', 'PullRequestEvent')
GROUP BY repo_name;
\end{lstlisting}