% Formatting of listings
\lstset{language=C, frame=L, basicstyle=\footnotesize,%\sffamily,
	keywordstyle=\bfseries, showstringspaces=false, xleftmargin=\parindent, numbers=none, numberstyle=\tiny, stepnumber=2, numbersep=14pt}
\newpage
\section{Проектирование}
\label{sec:Design}

В разделе 3.1. представлены функциональные и нефункциональные требования к системе. В разделе 3.2. описывается обработка данных. В разделе 3.3. представлено описание прогнозных моделей. В разделе 3.4. представлен описан модуль прогнозирования.

\subsection{Требования к системе}

Перечисление требований в зависимости от дальнейших выбранных библиотек и интеграций.

\subsection{Обработка данных}

 Для создания классификации проектов на GitHub необходимо понять, по каким критериям мы сможем выстроить и определить <<рейтинг>> популярных репозиториев. Для работы с информацией о проектах на GitHub будет использоваться существующий датасет, загруженный в ClickHouse, кроме того, он содержит достаточно много данных, которые могут быть полезны и иметь достаточно большой вес в возможности классифицировать репозиторий. Выделим основные параметры:
 \begin{itemize}
     \item Звезды (Stars). Количество звезд  репозитория указывает на его популярность. Звезды представляют интерес и поддержку сообщества разработчиков и пользователей. Большое количество звезд может привлечь внимание новых разработчиков и повысить репутацию репозитория.
     \item Форки (Forks). Количество форков показывает, сколько раз репозиторий был скопирован другими разработчиками для дальнейшей работы. Большое количество форков может означать активное участие сообщества, что не редко приводит к активному развитию.
    \item Проблемы (Issues). Количество задач отражает активность сообщества в обнаружении и решении проблем и задач. Активные задачи могут привлечь новых участников и улучшить качество проекта.
    \item Коммиты (Commits). Количество коммитов указывает на активность разработчиков в репозитории. Активность и последовательность коммитов важны для развития и поддержания проекта.
    \item Пулл-реквесты (Pull Requests). Количество запросов на слияние (pull requests) отражает вклад и сотрудничество участников проекта. Пулл-реквесты представляют собой важный инструмент совместной разработки и улучшения кода.
 \end{itemize}

 Текущий датасет содержит множество полей, где каждая запись соответствует определенной действия, выполненной пользователем в одном из репозиториев. В частности, существует столбец <<event\_type,>> который описывает различные операции. Для данной работы нас интересуют следующие операции: <<CommitCommentEvent>> (добавление коммита с комментарием), <<ForkEvent>> (создание форка репозитория), <<IssuesEvent>> (добавление обсуждения), <<PullRequestEvent>> (создание запроса на включение изменений), и <<WatchEvent>> (добавление звезды к проекту). В контексте работы с проектами, необходимо сгруппировать выполненные операции для каждого проекта. Следовательно, с использованием запросов можно получить информацию о названии проекта (репозитория) и количестве звезд, форков и других характеристик, связанных с данным репозиторием.

 После установления ключевых параметров, необходимо определить цель использования этих данных и методы их получения. В рамках данного исследования, нацеленного на разработку системы классификации проектов на платформе GitHub в зависимости от их популярности, имеется неотъемлемая потребность в доступе к данным из базы данных. Эти данные из датасета GitHub будут использованы для обучения модели классификации, её оценки и настройки. Оценка модели позволит установить, насколько успешно она способна разделять проекты на популярные и непопулярные.

Исходными данными для этой работы служит статья о наборе данных GitHub~\cite{clickHouse}, содержащем информацию о всех событиях на этой платформе с 2011 года и насчитывающем более трех миллиардов записей. Данный набор данных был загружен в ClickHouse, который представляет собой открытую систему управления базами данных, специально разработанную для эффективного анализа и хранения больших объемов информации. Одной из важных особенностей этой технологии является её акцент на аналитических задачах, что обеспечивает возможность проведения сложных анализов данных.

Текущий набор данных GH Archive представлен как в формате ClickHouse Native с объемом более 70 ГБ, так и в альтернативном формате, разделенном табуляцией, с объемом 85 ГБ. Учитывая, что данные из этого набора нужны лишь для однократного обучения модели без необходимости долгосрочного доступа ко всей базе данных, было принято решение извлекать такие объемы данных из облачных хранилищ, а не с физических устройств.

ClickHouse позволяет сохранять и получать данные из облачного хранилища ClickHouse.Cloud, однако на данный момент существуют ограничения, которые мешают получить доступ к этому ресурсу из-за территориальной основы. Следовательно, для продолжения работы отсутствует возможность воспользоваться данным сервисом.

Помимо вышеописанных методов извлечения данных, для демонстрации работы с общедоступными данными ClickHouse существует веб-страница, способная обрабатывать SELECT-запросы и предоставлять обширный объем информации. Все данные, полученные таким образом, являются полными и, следовательно, было принято решение осуществлять извлечение результатов запросов из HTML-страницы.

Для этой цели был разработан метод <<parsingHTML>>, который позволяет извлекать данные из разметочных файлов. Этот метод принимает параметр <<limit>>, который определяет желаемый объем данных для извлечения. Данный метод имитирует взаимодействие с веб-страницей, используя библиотеку Selenium WebDriver для управления браузером. Он создает виртуальное окружение браузера, отправляет запрос. После ожидания получения результата в течение 10 секунд с использованием библиотеки Beautiful Soup, предназначенной для анализа HTML-файлов, данные извлекаются из таблицы на веб-странице. Полученные данные могут быть представлены в формате CSV для дальнейшей обработки.

\subsection{Прогнозные модели}

Для более точного и достоверного результата необходимо правильно сформировать модели для дальнейшего анализа.

\subsection{Модуль прогнозирования}

Выбор наиболее подходящего алгоритма для прогнозирования в зависимости от приоритетной потребности.
