\newpage
\thispagestyle{empty}

\begin{adjustwidth}{-1.5cm}{0.5cm}
\begin{linespread}{1}
\begin{center}


\small{
МИНИСТЕРСТВО ОБРАЗОВАНИЯ И НАУКИ РОССИЙСКОЙ ФЕДЕРАЦИИ\\
Федеральное государственное бюджетное образовательное учреждение\\
высшего образования\\
\textbf{<<Южно-Уральский государственный университет\\
(национальный исследовательский университет)>>\\
Высшая школа электроники и компьютерных наук\\
Кафедра системного программирования}
}



\vspace{2em}

\hfill{}
\parbox{7cm}{
УТВЕРЖДАЮ \\
Зав. кафедрой СП \\[0.5em]
\underfield{} Л.Б.~Соколинский \\[0.5em]
06.02.2024
}

\vspace{2em}

\textbf{ЗАДАНИЕ} \\
% \parbox[t]{14cm}{
\textbf{на выполнение выпускной квалификационной работы бакалавра}\\
студентке группы КЭ-303\\
Гольденберг Дарье Игоревне,\\
обучающейся по направлению 09.03.04 <<Программная инженерия>> 
% }

\end{center}

\vspace{2em}

{
\small
\begin{enumerate}
	\bf\item Тема работы \rm \\
	Разработка системы для классификации проектов GitHub по популярности.
	\bf\item Срок сдачи студентом законченной работы: \rm
	05.06.2024.

	\bf\item Исходные данные к работе\rm
	\begin{enumerate}%[leftmargin=0.35cm]
		\raggedright
		\item Milovidov A., 2020. Everything You Ever Wanted To Know About GitHub (But Were Afraid To Ask), https://ghe.clickhouse.tech/
	\end{enumerate}

	\bf\item Перечень подлежащих разработке вопросов\rm
	\begin{enumerate}
		\item Провести анализ и спецификацию предметной области.
		\item Изучить статистику проектов, репозитории которых расположены на GitHub.
	\end{enumerate}

	\bf\item Дата выдачи задания: \rm
	09.02.2024.
\end{enumerate}

\vspace{1em}

\noindent
\textbf{Научный руководитель}\\[0.5em]
к.ф.-м.н., доцент
\hfill
\hbox to 8em{М.Л.~Цымблер\hfill}

\vspace{1em}

\noindent
\textbf{Задание принял к исполнению}
\hfill
\hbox to 8em{Д.И.~Гольденберг\hfill}

}

\thispagestyle{empty}

\end{linespread}
\end{adjustwidth}

\pagebreak
