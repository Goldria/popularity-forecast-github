    \newpage
\sectionnonumber{Введение}
\subsection*{Актуальность темы}
Актуальность работы заключается в том, что с развитием технологий тяжело понять, насколько может быть популярен проект, такая система может быть полезна для дальнейшей работы с репозиторием. 

\subsection*{Цель и задачи исследования}
Целью данной работы является разработка системы прогноза популярности проектов GitHub. В качестве исходных данных используются датасет ClickHouse о репозиториях на GitHub. Для достижения поставленной цели необходимо было решить следующие \textit{задачи}:
\begin{enumerateparen}
	\item провести анализ предметной области и обзор научной литературы по тематике исследования;
	\item разработать алгоритм обработки данных;
	\item 
\end{enumerateparen}
\vspace{0.5em}
\subsection*{Структура и объем работы}

Курсовая работа состоит из введения, пяти разделов, заключения и списка литературы. Объем работы составляет n страниц, объем библиографического списка составляет – k наименований.

\subsection*{Содержание работы}

\textbf{Первый раздел, «Обзор работ по тематике исследования»}, содержит обзор на работы по тематике исследования.

\textbf{Второй раздел, «Анализ предметной области»}, описывает постановку задачи и описание данных, которые будут использоваться для анализа.

\textbf{Третий раздел, «Проектирование»}, определяет требования к системе, описаны модели данных и структура приложения.

\textbf{Четвертый раздел, «Реализация»}, описывает реализацию компонентов системы.

\textbf{Пятый раздел, «Тестирование»}, описывает функциональное тестирование работы и эксперименты с разработанными моделями данных.

\textbf{В заключении} приведены основные итоги проделанной работы.